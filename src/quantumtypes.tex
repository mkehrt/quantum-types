\documentclass[times]{article}
\usepackage{mathpartir}
\usepackage{proof}

\title{Notes Towards a Quantum Type Theory}
\author{Matthew Kehrt}

\begin{document}

\newcommand{\prove}[3]{#1{\cdot}#2\vdash #3}

\maketitle
\section{Introduction}
This is meant to record my own thoughts on understanding quanum mechanics.  I have not looked at the literature on the state of analyzing quantum physics in terms of typed lambda calculi, but I know there is some prior work.

We first introduce some quantum logic rules, which indicate what actions are allowed to be performed on a quantum system.  The proof terms should be very simple, after that. 

\section{Quantum Logic Rules}

The judgment we use for reasoning about quantum systems is

\[ \prove{\Delta}{i}{\tau} \]

This is just a linear logic judgment decorated with a number representing probability amplitude.

\begin{mathpar}

\infer{\prove{\Delta_1;\Delta_2}{i}{\tau_1\otimes\tau_2}}{\prove{\Delta_1}{i}{\tau_1}{i} \\ \prove{\Delta_2}{i}{\tau_2}} \quad

\infer{\prove{\Delta}{i}{\tau_1\oplus\tau_2}}{\prove{\Delta}{i}{\tau_1} \\ \prove{\Delta}{i}{\tau_2}} \quad

\quad \&  

\end{mathpar}


\section{Proof Terms?}
\end{document}
